%Configuration options.
\documentclass[11pt,a4paper,oneside]{book}
\usepackage[utf8]{inputenc}
\usepackage[francais]{babel}
\usepackage{fontspec} % Required by french.
%\usepackage[T1]{fontenc}
\usepackage{mathtools}
\usepackage{amsfonts}
\usepackage{amssymb}
\usepackage[left=2cm,right=2cm,top=2cm,bottom=2cm]{geometry}

% Hypertext links.
\usepackage[hidelinks]{hyperref} % colorlinks=true

% This command is like \href except that it also inserts a foot note with the url printed, introduced by the third parameter.
\newcommand{\TSwiftUrl}[3]{\href{#1}{#2}\footnote{#3 : \url{#1}}}

\usepackage{longtable}
\usepackage{tabu}
%\usepackage{tikz}

\usepackage{graphicx}
\DeclareGraphicsExtensions{.pdf,.png,.jpg}

\usepackage[cache]{minted}
\usemintedstyle{xcode}

\author{Guillaume DIDIER}
\newcommand{\TSwiftTitle}[0]{Apprendre à programmer sur OS X en Swift avec Xcode}


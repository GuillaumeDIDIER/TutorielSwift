\chapter*{Préface}
\phantomsection
\addcontentsline{toc}{chapter}{Préface}
Vous avez un Mac, vous avez toujours rêver d'écrire des applications, ou d'apprendre à programmer ?
Ce tutoriel est pour vous.
Vous y apprendrez à programmer à partir de Zér0\footnote{Dédicace à feu le site du Zér0}  en utilisant le nouveau langage d'Apple, Swift, puis à écrire vos premières applications graphiques.

C'est mon premier gros cours, donc n'hésitez pas à faire remonter tout ce qui vous semble utile de rapporter.

Les Sources LaTeX et le "bugtracker" se trouvent \TSwiftUrl{https://github.com/GuillaumeDIDIER/TutorielSwift}{sur le dépot github ici}{Dépot Github du tutoriel}
Le PDF du cours est aussi disponible de façon provisoire \TSwiftUrl{https://onedrive.live.com/redir?resid=E2E36645B7EB4EE3!168&authkey=!AOH8IUaG-amgJ38&ithint=file\%2cpdf}{ici}{PDF du cours}.

En ce qui concerne OpenClassrooms, ce cours diffère du cours de Spader en cela qu'il se veut plus exigeant mais plus gratifiant, pour vous emmener plus loin, et plus rigoureux et exhaustif.
Je ne dis pas que son cours est moins bien, juste différent.

En étudiant ce cours
(désolé de vous dire que c'est du bouleau et pas de la salade)
vous acquerrez de véritables méthodes et compétences en programmation,
par le biais de nombreux exercices parsemé à travers ces pages.

Cher lecteur si tu es d'accord avec ce projet, accroches-toi et attention pour le décollage.

% Todo : Photo rafale PC au décollage.
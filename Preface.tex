\chapter*{Préface}
\phantomsection
\addcontentsline{toc}{chapter}{Préface}
Vous avez un Mac, vous avez toujours rêvé d'écrire des applications, ou d'apprendre à programmer ?
Ce tutoriel est pour vous.
Vous y apprendrez à programmer à partir de  Zér0\footnote{Dédicace à feu le site du Zér0} 
en utilisant le nouveau langage d'Apple, Swift, dans sa deuxième version, puis à écrire vos premières applications graphiques.

Pour suivre ce cours, il vous faut simplement ce qui suit.
\begin{itemize}
\item Posséder un Mac sous OS X 10.10 Yosemite ou plus récent,
\item Être persévérant, rigoureux et logique.
\item Ne pas faire une allergie à la langue de Shakespeare, ni à un peu de mathématiques (niveau collège).
\item Un peu de motivation !
\end{itemize}



C'est mon premier gros cours, donc n'hésitez pas à faire remonter tout ce qui vous semble utile de rapporter.

Les sources de la version LaTeX et le "bugtracker" se trouvent \TSwiftUrl{https://github.com/GuillaumeDIDIER/TutorielSwift}{sur le dépot github ici}{Dépot Github du tutoriel}
Le PDF du cours est aussi disponible de façon provisoire \TSwiftUrl{https://onedrive.live.com/redir?resid=E2E36645B7EB4EE3!168&authkey=!AOH8IUaG-amgJ38&ithint=file\%2cpdf}{ici}{PDF du cours}.

En étudiant ce cours
(désolé de vous dire que c'est un peu de bouleau et pas de la salade)
vous acquerrez de véritables méthodes et compétences en programmation,
par le biais de nombreux exercices parsemés à travers ces pages.

Cher lecteur si tu es d'accord avec ce projet, accroches-toi et attention pour le décollage.

\chapter{Application : Plus ? Ou moins ?}
\section{Votre premier TP}
Ce chapitre ne contient pas d'informations nouvelles mais est entièrement dédié à la pratique. Je vais vous donner ici un sujet, accompagné de conseil et vous laisser chercher, avant de vous laisser regarder la correction. Ce TP est une dédicace au premier tuto de programmation que j'ai suivi, l'énoncé original se trouve ici.
Comme je l'ai déjà dis, la pratique est essentielle à la réussite, donc ne vous précipitez pas sur la correction, et acceptez de devoir chercher un peu. En plus, une fois que l'on a trouvé soi-même la solution et que l'on voit son programme tourner, c'est une vraie satisfaction personnelle. Soyez persévérant, il vous faudra peut être plusieurs essais pour trouver une solution. Si vous ne comprenez pas pourquoi ça ne fait pas ce que vous croyez par contre, n'hésitez pas à poser des questions; mais ne regardez pas la solution avant d'avoir quelque chose. Prenez votre temps, je ne suis pas en train de vous dire "Il vous reste un [sic] minute" en prenant un air sadique. (Dédicace à nos surveillants de prépa, qui n'était pas sadique d'ailleurs.)
Personnellement j'utilise une approche du plus haut niveau vers le bas niveau, en commençant par faire un plan, un squelette que je viens compléter.
\section{Consignes}
Votre mission, si vous l'acceptez, (enfin vous n'avez pas le choix ! ) est d'écrire.
\section{Indications}
\section{Correction}
\section{Pour aller plus loin}

\chapter{Les Boucles}
Dans ce chapitre, nous allons voir comment
répéter des instructions un nombre variable de fois.
\section{La boucle While}
Cette première boucle permet de répéter un groupe d'instruction tant qu'une condition est vérifié (ou que sa négation n'est pas vérifiée.).

\begin{listing}[h]
\begin{minted}[linenos]{swift}
while <condition> {
    // code à exécuter.
}
\end{minted}
\caption{Syntaxe de la boucle While.}
\end{listing}

Par exemple :
\begin{listing}[h]
\begin{minted}[linenos]{swift}
while !l_utilisateur_veut_quitter {
    // Continuer l'application.
}
// L'utilisateur veut quitter.
// On fait le ménage et on quitte.
\end{minted}
\caption{Syntaxe de la boucle While.}
\end{listing}

\section{Énumérons des entiers, la boucle for in}
La deuxième syntaxe de boucle que nous allons étudier permet d'énumérer des entiers.
La première ligne de la boucle vous permet de nommer une variable qui prendra successivement toute les valeurs entière de la plage d'entier que vous lui fournirez.
Cette variable n'est ensuite accessible que dans le corps de la boucle,
contenu entre deux accolades.
Si elle était déja définie,
la seconde définition masquera la première dans le corps de la boucle,
et la première sera de nouveau accessible après la fin de la boucle.
Cependant ce code, qui semble toléré par le compilateur,
n'est pas recommandé, et je vous le déconseille très fortement.
\subsection{Les plages d'entiers, ..< et ...}
Ce sont deux opérateurs permettant d'obtenir une suite d'entiers entre deux bornes. \verb"a..<b" correspondra aux entiers entre a inclus et b exclus, tandis que \verb"a...b" correspondra à a et b inclus. Ils renvoient tous deux un objet de type range, type que nous n'allons pas traiter ici, mais dont la boucle for extraira à chaque itération un entier, jusqu'à ce qu'il n'y en aie plus, ce qui arrêtera la boucle.
\paragraph{Attention :}
Il ne faut pas que b < a, sinon vous obtenez une erreur, voir une boucle infinie. 
\subsection{La syntaxe complète}
\begin{listing}[h]
\begin{minted}[linenos]{swift}
let n /*: Int*/ = <ce que vous voulez>
for i/* nom de variable valide*/ in 0..<n /*ou a...b,
comme vous voulez */{
    // Corps de la boucle à exécuter
    // i prend successivement toutes les valeurs contenues dans le Range.
    println(i) // par exemple.
}
// i n'est plus accessible
\end{minted}
\caption{Syntaxe de la boucle for in}
\end{listing}
\paragraph{Remarque :}
La boucle for in à en fait une application plus générale, mais nous en parlerons par la suite. %TODO ref
\subsection{Exercice}
Essayez d'écrire une boucle while équivalente.

\section{Seconde syntaxe de la boucle for}
Cette seconde syntaxe était la première historiquement,
il s'agit en effet de la syntaxe historique de cette boucle en C,
à ceci près que la version C exige des parenthèse
qu'il n'est pas nécessaire de mettre en Swift.

\begin{listing}[h]
\begin{minted}[linenos]{swift}
for /initialisation, eg*/ var i = 0; /*Condition*/ i<100;
/*Action à exécuter après chaque itération eg incrémentation*/ i++{
    // Corps de la boucle à exécuter, tant que la condition n'est pas vérifié.
    println(i) // par exemple.
}
// Une variable définie juste après for n'est plus disponible.
\end{minted}
\caption{Syntaxe de la boucle for in}
\end{listing}
Ici l'initialization déclare une variable,
mais ce n'est pas nécessaire.
L'initialisation est une instruction quelconque, éventuellement vide.
La condition est une condition, qui peut être vide, ce qui causerai une boucle infinie.
La dernière instruction ne doit pas être vide.
\section{Application : Copions des lignes}
\subsection{Consignes}
Votre but est d'écrire un programme qui
écrive un certain nombre de fois fixé dans une constante $n$ une ligne telle que << \emph{Je ne lancerai pas d'avions en papier.} >>.
\pagebreak % Let's not cheat.
\subsection{Indications}
\begin{itemize}
\item Il vous faut une boucle, mais de quel type ?

\item Stocker la ligne et le nombre de fois dans deux constantes.

\item ...\ est votre ami.
\end{itemize}
\pagebreak % Let's not cheat.
\subsection{Correction}
Posez vos stylos !
\begin{listing}[h]
\begin{minted}[linenos]{swift}
let n : UInt = 50 // Je ne suis pas complètement fou.
let punition : String = "Je dois chercher mes exercices !"
// Je ne vise personne ...
for i in 1...n { // n inclus
    println("\(i). \(punition)")
}
\end{minted}
\caption{Correction des lignes copiées.}
\end{listing}%\) Against stupid syntactique coloration.
\section{Altérer l'exécution des boucles}
\paragraph{Attention :}
L'abus de ces fonctionnalités est nuisible à la santé de qui lira votre code.

\chapter{Les Boucles}
Dans ce dernier chapitre avant votre premier chapitre de TP,
nous allons voir comment répéter des instructions un nombre variable de fois.
\section{Les boucles while et do-while}
\subsection{Une boucle ?}

\subsection{while}
Cette première boucle permet de répéter un groupe d'instruction tant qu'une condition est vérifié (ou que sa négation n'est pas vérifiée.).

\begin{listing}[H]
\begin{minted}[linenos]{swift}
while <condition> {
    // code à exécuter.
}
\end{minted}
\caption{Syntaxe de la boucle While.}
\end{listing}

Par exemple :
\begin{listing}[h]
\begin{minted}[linenos]{swift}
while !l_utilisateur_veut_quitter {
    // Continuer l'application.
}
// L'utilisateur veut quitter.
// On fait le ménage et on quitte.
\end{minted}
\caption{Syntaxe de la boucle While}
\end{listing}
\subsection{do-while}
Cette boucle est similaire à la précédente, à ceci près que le code contenu dans la boucle est toujours exécuté au moins une fois, et que la conditions est placé (et donc évaluée, à la fin de la boucle).

Comme l'utilisateur ne vas certainement pas essayer de quitter avant d'avoir utilisé le programme, on peut réécrire le code de l'exemple précédent comme ceci :
\begin{listing}[H]
\begin{minted}[linenos]{swift}
// Initialisation du programme
do {
    // La boucle principale,
    // exécutée tant que l'utilisateur ne veut pas quitter.
} while !l_utilisateur_veut_quitter
// L'utilisateur veut quitter.
// On fait le ménage et on quitte.
\end{minted}
\caption{Avec une boucle Do While}
\end{listing}

\section{Énumérons des entiers, la boucle for in}
La deuxième syntaxe de boucle que nous allons étudier permet d'énumérer des entiers.
La première ligne de la boucle vous permet de nommer une variable qui prendra successivement toute les valeurs entière de la plage d'entier que vous lui fournirez.
Cette variable n'est ensuite accessible que dans le corps de la boucle,
contenu entre deux accolades.
Si elle était déja définie,
la seconde définition masquera la première dans le corps de la boucle,
et la première sera de nouveau accessible après la fin de la boucle.
Cependant ce code, qui semble toléré par le compilateur,
n'est pas recommandé, et je vous le déconseille très fortement.
\subsection{Les plages d'entiers, ..< et ...}
Ce sont deux opérateurs permettant d'obtenir une suite d'entiers entre deux bornes. \verb"a..<b" correspondra aux entiers entre a inclus et b exclus, tandis que \verb"a...b" correspondra à a et b inclus. Ils renvoient tous deux un objet de type range, type que nous n'allons pas traiter ici, mais dont la boucle for extraira à chaque itération un entier, jusqu'à ce qu'il n'y en aie plus, ce qui arrêtera la boucle.
\paragraph{Attention :}
Il ne faut pas que b < a, sinon vous obtenez une erreur, voir une boucle infinie. 
\subsection{La syntaxe complète}
\begin{listing}[h]
\begin{minted}[linenos]{swift}
let n /*: Int*/ = <entier positif>
for i/* nom de variable valide*/ in 0..<n /*ou a...b,
comme vous voulez */{
    // Corps de la boucle à exécuter
    // i prend successivement toutes les valeurs contenues dans le Range.
    println(i) // par exemple.
}
// i n'est plus accessible
\end{minted}
\caption{Syntaxe de la boucle for in}
\end{listing}
\subsection{Exercice}
Essayez d'écrire une boucle while équivalente.

\subsection{La boucle for et les chaines de caractères}



\section{Seconde syntaxe de la boucle for}
Cette seconde syntaxe était la première historiquement,
il s'agit en effet de la syntaxe historique de cette boucle en C,
à ceci près que la version C exige des parenthèse
qu'il n'est pas nécessaire de mettre en Swift.

\begin{listing}[h]
\begin{minted}[linenos]{swift}
for /initialisation, eg*/ var i = 0; /*Condition*/ i<100;
/*Action à exécuter après chaque itération eg incrémentation*/ i++{
    // Corps de la boucle à exécuter, tant que la condition n'est pas vérifié.
    println(i) // par exemple.
}
// Une variable définie juste après for n'est plus disponible.
\end{minted}
\caption{Syntaxe de la boucle for in}
\end{listing}
Ici l'initialization déclare une variable,
mais ce n'est pas nécessaire.
L'initialisation est une instruction quelconque, éventuellement vide.
La condition est une condition, qui peut être vide, ce qui causerai une boucle infinie.
La dernière instruction ne doit pas être vide.
\section{Application : Copions des lignes}
\subsection{Consignes}
Votre but est d'écrire un programme qui
écrive un certain nombre de fois fixé dans une constante $n$ une ligne telle que << \emph{Je ne lancerai pas d'avions en papier.} >>.
\pagebreak % Let's not cheat.
\subsection{Indications}
\begin{itemize}
\item Il vous faut une boucle, mais de quel type ?

\item Stocker la ligne et le nombre de fois dans deux constantes.

\item ...\ est votre ami.
\end{itemize}
\pagebreak % Let's not cheat.
\subsection{Correction}
Posez vos stylos !
\begin{listing}[h]
\begin{minted}[linenos]{swift}
let n : UInt = 50 // Je ne suis pas complètement fou.
let punition : String = "Je dois chercher mes exercices !"
// Je ne vise personne ...
for i in 1...n { // n inclus
    println("\(i). \(punition)")
}
\end{minted}
\caption{Correction des lignes copiées.}
\end{listing}%\) Contre la coloration syntactique débile.
\section{Altérer l'exécution des boucles}
\paragraph{Attention :}
L'abus de ces fonctionnalités est nuisible à la santé de qui lira votre code.

\subsection{break}
Cette instruction, située dans un switch ou une boucle permet de sortir des accolades, en sautant tout le code qui pouvait éventuellement rester dans les accolades.

Cela revient à sauter à l'accolade fermante suivante de switch ou de boucle \emph{la plus proche}.
\paragraph{Attention :}N'utilisez cette syntaxe pour sortir d'une boucle while que pour des cas très particulier, qui ne peuvent être exprimé par une condition. En effet cela nuit \emph{grandement} à la lisibilité de votre code. (Certains de mes professeurs considère que break est le mal absolue (comme l'instruction goto, bannie des langages les plus modernes pour la même raison, en pire.)
\subsubsection{Syntaxe basique}

\subsubsection{Les "labels" de boucle et de switch}
Il est possible de mettre un nom sur une boucle ou un switch, et en précisant ce nom de sortir de ce bloc de l'intérieur. Ce ci permet de sortir de plusieurs blocs d'un seul coup, comme par exemple sortir d'une boucle contenant un switch depuis l'intérieur de ce switch.
\subsection{continue}

\section{Retour sur Switch}
Je reviens commme promis sur les switch pour vous présenter certaines des fonctionnalités avancées que je ne vous ai pas présenté au chapitre précédent.

\subsubsection{Capturer un intervalle} 
Il est en fait possible de faire un peu plus avec un switch :
Tout d'abord il est possible de traiter un intervalle d'un coup, en utilisant la notation d'intervalle :
\begin{listing}{h}
\begin{minted}[linenos]{swift}
switch(un_nombre){
case 1..5 :
    // Code exécuté pour toute valeur entre 1 et 5

}
\end{minted}
\caption{Un intervalle}
\end{listing}
\subsubsection{Un switch sur plusieurs valeurs}
Je détaille ici une syntaxe un peu plus complexe, qui fait intervenir en réalité un type que nous verrons par la suite.

\subsubsection{Capturer la valeur d'une variable dans un case}

\begin{listing}[h]
\begin{minted}[linenos]{swift}
switch(un_nombre){
case let x :
    // Code exécuté pour toute valeur, x prenant cette valeur.
    // Ceci a peu d'intérêt

}

switch (x, y){
case (let abscisse, 0)
    println("Point d'abscisse \(abscisse) sur cet axe.")
default:
    break
}

\end{minted}
\caption{Capturer une valeur}
\end{listing}%\) Contre la coloration syntactique débile.
Cela n'a pas d'intérêt ici, mais si vous obtenez les deux valeurs d'un coup dans une seule variable, cela présente un intérêt. Je reviendrai sur ce type plus loin.
\subsection{fallthrough}
Il est possible de poursuivre l'exécution d'un case dans le case suivant, en utilisant le mot clé \mintinline{swift}{fallthrough}.
\chapter{Les variables et constantes}
Comme je l'ai dit, un ordinateur manipule des données.
Il faut donc pouvoir les stocker quelque part.
\section{La mémoire d'un ordinateur, qu'est-ce qu'une \emph{variable} ?}
Une \emph{variable},
c'est avant tout un compartiment de la mémoire de l'ordinateur 
sur lequel on colle un nom,
et dans lequel on a le droit de mettre ce que l'on veut,
pour pouvoir y accéder ultérieurement.

L'unité de base de la mémoire d'un ordinateur est \emph{le} bit (pas la !),
qui possède deux états, 1 ou 0,
à comparer aux 10 chiffres de notre système décimal.

Pour stocker des nombres
plus grand, il faut mettre plusieurs bits cote à cote,
comme nous mettons plusieurs chiffres cote à cote pour écrire 1999, par exemple.
Dans ce cas, le dernier chiffre correspond aux unités, l'avant dernier à
des paquets de 2, l'antépénultième 4, ... 8, 16, 32, 64... .

Les bits sont regroupés en mots appelés \emph{bytes} en anglais,
d'une taille fixée par l'architecture du processeur.
En français on traduit improprement ce terme par octet,
qui désigne un mot/byte de 8 bit, pouvant prendre
\begin{math}2*2*2...*2\end{math} huit fois, noté \begin{math} 2^{8} \end{math},
soit 256 valeurs (retenez ce nombre, on le retrouvera par la suite.)
Toute la mémoire est découpé en bytes.
On manipule forcément un ensemble de byte (1, 2, 4, ...),
et donc toute variable a une taille définie qui est un multiple de l'octet.

L'ordinateur peut aussi stocker des nombres binaires à virgule,
comme nos nombres à virgule, avec les limitations de ces derniers.
Je ne détaille pas ici toute la représentation, si vous voulez des détails,
je vous renvoie à cet excellent \TSwiftUrl{https://zestedesavoir.com/forums/sujet/3083/fonctionnement-des-ordinateurs-de-zero/}{tutoriel, actuellement en bêta}{Fonctionnement d'un ordinateur à partir de zéro --- Zeste de Savoir, actuellement en bêta}.

Sachez toutefois que
par exemple
$0.1$ n'est pas codable exactement avec un nombre fini de bit,
tout comme \(\frac{1}{3}\) en numération décimale.


\emph{Mais il n'y a que des nombres, comment on stocke du texte ?!}
Désolé, mais pour l'ordinateur, tout est un nombre,
tout dépend de la façon dont on l'interprète,
on en viendra au texte un peu plus tard.

\section{Demander poliment de la place pour stocker un nombre}
Passons à la pratique !

\subsection{Déclarons l'age du Capitaine}

\begin{listing}[h]
\caption{Premier exemple, l'âge du capitaine}
\begin{minted}[linenos=true]{swift}
var age_du_capitaine = 42 // ;-)
\end{minted}
\end{listing}

Décortiquons ce qui précède :
\begin{description}
\item[var :] mot-clé indiquant que l'on veut déclarer une variable.
\item[age\_du\_capitaine :] nom de la variable,
qui nous permettra d'y accéder de nouveau.
\emph{Attention, il ne peut pas y avoir deux variables de même nom}
\item[= :] opérateur utilisé pour donner une valeur à une variable.
\item[42 :] la valeur entière que l'on donne à notre variable.
\item[// ;-) :] 'est un commentaire. Si vous ne le saviez pas, \emph{relisez le chapitre précédent}.
\end{description}
\paragraph{Attention :}
Ne mettez pas de zéro suivi de chiffres non nuls,
car vous auriez des surprises (octales).
Je reviendrait la dessus dans un chapitre ultérieur.
\subsection{Différents types d'entiers}
Je vous ai parlé de la représentation en mémoire des entiers,
donc vous devriez me demander comment Swift détermine
le nombres d'octets qui sont attribués à notre entier.
La réponse c'est qu'à moins qu'on le lui précise,
il prend une valeur par défaut.
Il existe plusieurs types d'entiers,
chacun pouvant contenir une plage de nombre particulière,
selon le nombre d'octets qu'il comprennent.
Notamment, il existe des variantes non signées,
qui ne contiennent que des nombres positifs.

\begin{longtabu} to \linewidth {|X[3,l,m]|X[1.5,l,m]|X[6,r,m]|X[1,r,m]|}
\hline Taille & Type & Plage de valeur & Précision \\ \hline
\endhead
Architecture (32 ou 64 bits) & \mintinline{swift}{Int} (par défaut) & 32 bits : \mintinline{swift}{Int32}, 64 bits : \mintinline{swift}{Int64} & 1 \\ \hline
Architecture (32 ou 64 bits) & \mintinline{swift}{UInt} & 32 bits : \mintinline{swift}{UInt32}, 64 bits : \mintinline{swift}{UInt64} & 1 \\ \hline
1 octet soit 8 bits & \mintinline{swift}{Int8} & -128 à 127 & 1 \\ \hline
1 octet soit 8 bits & \mintinline{swift}{UInt8} & 0 à 255 & 1 \\ \hline
2 octets soit 16 bits & \mintinline{swift}{Int16} & -32768 à 32767 & 1 \\ \hline
2 octets soit 16 bits & \mintinline{swift}{UInt16} & 0 à 65535 & 1 \\ \hline
4 octets soit 32 bits & \mintinline{swift}{Int32} & -2147483648 à 2147483647 & 1 \\ \hline
4 octets soit 32 bits & \mintinline{swift}{UInt32} & 0 à 4294967295 & 1 \\ \hline
8 octets soit 64 bits & \mintinline{swift}{Int64} & -9223372036854775808 à 9223372036854775807 & 1 \\ \hline
8 octets soit 64 bits & \mintinline{swift}{UInt64} & 0 à 18446744073709551615 & 1 \\ \hline

\caption{Les différents types d'entiers.}
\end{longtabu}

\paragraph{Attention :}
N'essayez pas de mélanger des types différents ou
le compilateur vous le fera (amèrement) regretter.
On discutera de ce problème un peu plus en détail dans la quatrième section.

Par exemple, pour déclarer que l'on stocke l'âge du capitaine sur 1 octet
(soit 8 bits) et que ce doit être un entier non signé (positif), on utilise le code suivant.

\begin{listing}[h]
\caption{Un type plus approprié pour l'âge du capitaine}
\begin{minted}[linenos]{swift}
var age_du_pitaine : UInt8  = 42 // Question : quel est l'âge maximal du capitaine ?
\end{minted}
\end{listing}
Le caractère \verb":" permet d'introduire une \emph{annotation de type},
tandis que \verb"UInt8" correspond au type d'entiers choisi.

Remarquez que Swift est ici plutôt agréable à utiliser puisqu'il ne nous oblige pas à lui préciser le type lorsque la réponse est évidente, mais nous permet d'être plus précis si on en a besoin.
\paragraph{L'age maximal du capitaine est :}
255 ans.

C'est parfaitement approprié, puisque c'est le type le plus proche
qui soit \emph{non signé (positif)}, ce qui est logique (car qui connaît un capitaine ayant -10 ans ?), et qui aie la plage la moins grande possible contenant toute les valeurs possibles (de zéro à environ une grosse centaine d'années).
Un entier signé sur 8 bits aurait posé problème
car on connait des humains ayant 130 ans.

Enfin, cela suppose que le capitaine soit un humain
et pas un être ayant une plus grande longévité, un elfe par exemple.

\subsection{La structure générale}
La structure générale est la suivante,
où les parties entre crochets peuvent être absentes,
mais pas toutes les deux en même temps.

\begin{listing}[h]
\begin{minted}[linenos]{swift}
var nom_de_variable> [ : Type ] [ = valeur_initiale ]
\end{minted}
\caption{Structure générale d'une déclaration de variable.}
\end{listing}

\begin{description}

\item[nom de variable :]
un nom de variable peut être composé de n'importe quel caractère Unicode
à part \emph{les espaces, les flèches et symboles mathématiques et les
caractères Unicode invalides ou servant à dessiner des lignes et des boites},
soit à peu près tout ce que vous pouvez vouloir utiliser.
Il ne doit pas commencer par un chiffre, ni correspondre à un mot clé du langage.


\item[Type :] il peut être omis si une valeur initiale est précisée,
permettant à Swift de déduire le type.
Le mot clé \mintinline{swift}{typealias} permet de définir un type
comme étant la même chose qu'un type déjà existant.
(Les noms de types suivent les même règles que les noms de variable).
Exemple :
\mintinline{swift}{typealias Age = UInt8}
Par convention les noms de type sont en CamelCase,
c'est à dire fait de mots dont la première lettre est en majuscule
collés les uns aux autres sans espaces.
\emph{Toute variable possède un type.}

\item[valeur initiale :]
il est recommandé d'initialiser une variable à sa déclaration,
mais ce n'est pas obligatoire.
Dans ce cas, il est indispensable de préciser le type pour que Swift s'y retrouve. 
Il faut alors aussi garantir que la variable sera toujours initialisée
avant d'y accéder pour la première fois, sinon le programme ne compilera pas.
Attention, le compilateur Swift est très strict là dessus.
\end{description}

Il est possible de déclarer plusieurs variables en même temps, comme dans cet exemple.
\mint{swift}|var frères = 5, sœurs = 4 // c'est une famille nombreuse !|
\subsection{Mots-clés}

%TODO : Déplacer dans les références.
\begin{itemize}

\item Mots-clés utilisés dans les déclarations :
\mintinline{swift}{class},
\mintinline{swift}{deinit},
\mintinline{swift}{enum},
\mintinline{swift}{extension},
\mintinline{swift}{func},
\mintinline{swift}{import},
\mintinline{swift}{inout}
\mintinline{swift}{init},
\mintinline{swift}{internal},
\mintinline{swift}{let},
\mintinline{swift}{operator},
\mintinline{swift}{private},
\mintinline{swift}{protocol},
\mintinline{swift}{public},
\mintinline{swift}{static},
\mintinline{swift}{struct},
\mintinline{swift}{subscript},
\mintinline{swift}{typealias} et
\mintinline{swift}{var}

\item Mots-clés utilisés dans des instructions :
\mintinline{swift}{break},
\mintinline{swift}{case},
\mintinline{swift}{continue},
\mintinline{swift}{default},
\mintinline{swift}{defer}
\mintinline{swift}{do},
\mintinline{swift}{else},
\mintinline{swift}{fallthrough},
\mintinline{swift}{for},
\mintinline{swift}{guard},
\mintinline{swift}{if},
\mintinline{swift}{in},
\mintinline{swift}{repeat},
\mintinline{swift}{return},
\mintinline{swift}{switch},
\mintinline{swift}{where} et
\mintinline{swift}{while}

\item Mots-clés utilisés dans des expressions et types :
\mintinline{swift}{as},
\mintinline{swift}{catch},
\mintinline{swift}{dynamicType},
\mintinline{swift}{false},
\mintinline{swift}{is},
\mintinline{swift}{nil},
\mintinline{swift}{rethrows},
\mintinline{swift}{self},
\mintinline{swift}{Self},
\mintinline{swift}{super},
\mintinline{swift}{throw},
\mintinline{swift}{throws},
\mintinline{swift}{true},
\mintinline{swift}{try},
\mintinline{swift}{__COLUMN__},
\mintinline{swift}{__FILE__},
\mintinline{swift}{__FUNCTION__} et
\mintinline{swift}{__LINE__}

\item Mot clé utilisé dans des \emph{patterns}:
\mintinline{swift}{_}

\item Mots-clés utilisés dans des contextes particuliers (en dehors de ces contextes, ils ne sont pas considérés comme des mots-clés,
mais dans le doute je vous recommande de ne pas les utiliser) :
\mintinline{swift}{associativity},
\mintinline{swift}{convenience},
\mintinline{swift}{dynamic},
\mintinline{swift}{didSet},
\mintinline{swift}{final},
\mintinline{swift}{get},
\mintinline{swift}{infix},
\mintinline{swift}{inout},
\mintinline{swift}{lazy},
\mintinline{swift}{left},
\mintinline{swift}{mutating},
\mintinline{swift}{none},
\mintinline{swift}{nonmutating},
\mintinline{swift}{optional},
\mintinline{swift}{override},
\mintinline{swift}{postfix},
\mintinline{swift}{precedence},
\mintinline{swift}{prefix},
\mintinline{swift}{Protocol},
\mintinline{swift}{required},
\mintinline{swift}{right},
\mintinline{swift}{set},
\mintinline{swift}{Type},
\mintinline{swift}{unowned},
\mintinline{swift}{weak} et
\mintinline{swift}{willSet}
\end{itemize}

\subsection{Les nombres à virgule flottante}

Parlons maintenant des nombres à virgule (ou à point, puisque tel est le séparateur décimal en Angleterre et donc dans le monde des programmeurs;-).

Ces nombres ont un  défaut :
ils ne peuvent pas représenter parfaitement tous les nombres décimaux,
comme un nombre décimal ne peut pas représenter parfaitement
\begin{math} \frac{1}{7} \end{math}, par exemple.
Mais en général cette approximation n'est pas un problème.
Il y deux principaux types de nombres à virgule, ainsi qu'un troisième,
non documenté, que je cite par soucis d'exhaustivité :
\begin{longtabu} to \linewidth {|X[3,l,m]|X[2,l,m]|X[4,r,m]|X[3,r,m]|}
\hline Taille & Type & Plage de valeur (approximative) & Précision \\ \hline
\endhead
4 octets soit 32 bits & \mintinline{swift}{Float} & \begin{math}\pm10^{-38}\end{math} à \begin{math}\pm10^{-38}\end{math} & 6 chiffres décimaux \\ \hline % Vérifier la précision.
8 octets soit 64 bits & \mintinline{swift}{Double} & \begin{math}\pm10^{-308}\end{math} à \begin{math}\pm10^{308}\end{math} & 15 chiffres décimaux \\ \hline
10 octets soit 80 bits & \mintinline{swift}{Float80} & \begin{math}\pm 3.65×10^{−4951}\end{math} à \begin{math}\pm1.18×10^{4932}\end{math} & 18 chiffres décimaux \\ \hline

\caption{Les différents types de nombres à virgule flottante}
\end{longtabu}

Il est aussi possible d'insérer des \verb"_" entre les chiffres,
pour formater ses nombres de manière plus (ou moins, au choix) lisible.

\subsection{Changer la valeur d'une variable}
Pour changer la valeur d'une variable, il suffit d'utiliser l'opérateur \verb"=".
\begin{listing}[h]
\caption{Changer la valeur d'une variable.}
\begin{minted}[linenos]{swift}
var sens_de_la_vie = 5 // Int.
sens_de_la_vie = 42 // Parce que c'est la seule bonne réponse !
var six_fois_sept = sens_de_la_vie // Ça marche aussi.
var pi = 3.14 // Double
pi = 3.141592 // Plus précis !
\end{minted}
\end{listing}

\emph{Attention, on ne peut pas changer le type d'une variable après qu'elle a été défini.}
\subsection{Récapitulatif des types de nombres}
\begin{longtabu} to \linewidth {|X[3,l,m]|X[1,l,m]|X[4,r,m]|X[2,r,m]|}
\hline Taille & Type & Plage de valeur (approximative) & Précision \\ \hline
\endhead
Architecture (32 ou 64 bits) & \mintinline{swift}{Int} (par défaut) & 32 bits : \mintinline{swift}{Int32}, 64 bits : \mintinline{swift}{Int64} & 1 \\ \hline
Architecture (32 ou 64 bits) & \mintinline{swift}{UInt} & 32 bits : \mintinline{swift}{UInt32}, 64 bits : \mintinline{swift}{UInt64} & 1 \\ \hline
1 octet soit 8 bits & \mintinline{swift}{Int8} & -128 à 127 & 1 \\ \hline
1 octet soit 8 bits & \mintinline{swift}{UInt8} & 0 à 255 & 1 \\ \hline
2 octets soit 16 bits & \mintinline{swift}{Int16} & -32768 à 32767 & 1 \\ \hline
2 octets soit 16 bits & \mintinline{swift}{UInt16} & 0 à 65535 & 1 \\ \hline
4 octets soit 32 bits & \mintinline{swift}{Int32} & -2147483648 à 2147483647 & 1 \\ \hline
4 octets soit 32 bits & \mintinline{swift}{UInt32} & 0 à 4294967295 & 1 \\ \hline
8 octets soit 64 bits & \mintinline{swift}{Int64} & -9223372036854775808 à 9223372036854775807 & 1 \\ \hline
8 octets soit 64 bits & \mintinline{swift}{UInt64} & 0 à 18446744073709551615 & 1 \\ \hline

4 octets soit 32 bits & \mintinline{swift}{Float} & \begin{math}\pm10^{-38}\end{math} à \begin{math}\pm10^{-38}\end{math} & 6 chiffres décimaux \\ \hline % Vérifier la précision.
8 octets soit 64 bits & \mintinline{swift}{Double} & \begin{math}\pm10^{-308}\end{math} à \begin{math}\pm10^{308}\end{math} & 15 chiffres décimaux \\ \hline
10 octets soit 80 bits & \mintinline{swift}{Float80} & \begin{math}\pm 3.65×10^{−4951}\end{math} à \begin{math}\pm1.18×10^{4932}\end{math} & 18 chiffres décimaux \\ \hline

\caption{Les différents types de nombres}
\end{longtabu}
Notez aussi que les nombres littéraux que vous tapez n'ont pas de type a priori, et que le type résultant n'est déterminé qu'après comme étant Int, ou Double, si le résultat n'est pas entier. Il existe plusieurs notations (décimale, hexadécimale, octale, binaire), mais je ne vous les présente pas ici.
\section{Variables et Constantes}
On peut en fait imposer que la valeur d'une variable ne change plus après sa création :
c'est alors une constante. Elle se déclare comme une variable excepté que l'on utilise le
mot clé \mintinline{swift}{let}, et que sa valeur doit être fixé à la déclaration, dans ce cas toute tentative de changement de la valeur résulte en une erreur à la compilation.
\begin{listing}[h]
\begin{minted}[linenos=true]{swift}
let pi = 3.141592
// Que personne n'essaye de changer pi !
pi = 3.1416 // Erreur !
\end{minted}
\end{listing}

Du fait de cette déclaration, le compilateur peut d'une part vous prémunir d'une erreur, et
d'autre part réaliser des optimisations supplémentaires. Apple vous recommande d'utiliser
des constantes en priorité, et de ne déclarer comme variable que ce qui doit l'être.
\section*{Conclusion}
\phantomsection
\addcontentsline{toc}{section}{Conclusion}
Dans cette partie vous avez appris :
\begin{itemize}
\item ce qu'était une variable.
\item qu'il existe plusieurs types de variables.
\item comment déclarer une variable entière
\item comment déclarer un nombre à virgule.
\item comment changer la valeur d'un variable.
\item ce qu'est une constante.
\end{itemize}

\section*{Exercice}
\phantomsection
\addcontentsline{toc}{section}{Exercice}
\begin{enumerate}
\item Déclarez une variable de type approprié pour contenir votre age, ainsi qu'une variable pour votre taille en mètres, et initialisez les avec leur valeurs de il y a un an.
\item Mettez à jour les deux variables avec les valeurs d'aujourd'hui.
\item Déclarez une constante pour votre année de naissance.
\end{enumerate}

\chapter{Les conditions}
Après le chapitre précédent assez théorique, ce chapitre vous permettra enfin de réaliser des programmes un peu plus intéressants.
\section{Si… Alors… Sinon …}
\emph{<< C'est bien joli, les booléens, mais à quoi servent-ils ? >>}
Une première utilisation des booléens est le <<\emph{branchement conditionnel}>>.
\subsection{Exécuter des instructions uniquement lorsque une condition est vérifiée}
On va prendre un exemple : on suppose que vous avez obtenu de l'utilisateur deux nombres $a$ et $b$, stockés dans deux variables, pour calculer le quotient $\frac{a}{b}$.
Cependant, si $b$ est nul, vous ne pouvez pas effectuer la division. %Not true anymore (En fait en Swift, avec des nombres à virgule vous obtiendrez $\pm \infty$)

On se propose de vérifier si $b$ est nul et d'écrire un message à la console dans ce cas.

Grâce au chapitre précédent, vous pouvez me trouver le condition correspondante.

\pagebreak % No cheating
\mintinline{swift}{b == 0}, comme vous avez tous trouvé (sans tricher).

Pour exécuter \mintinline{swift}{println("b est nul ! Division impossible.")}, nous allons utiliser notre première instruction composée : \mintinline{swift}{if}.

\begin{listing}[h]
\begin{minted}[linenos]{swift}
let b = 0

// /!\ Abus de commentaire

if b == 0 /*Condition*/ { // Condition non nécessairement entre parenthèses
    // Code à exécuter si la conditions est vérifié.
    // Éventuellement plusieurs instructions.
    println("b est nul ! Division impossible.")
} // Les accolades ouvrantes et fermantes pourraient respectivement
  // être mise une ligne après le if,
  // ou sur la même ligne que la dernière instruction.
// Code exécuté dans tout les cas.
\end{minted}
\caption{la syntaxe de if.}
\end{listing}
\paragraph{Deux commentaires pour ceux qui ont déjà programmé
dans un langage utilisant la syntaxe du C.}
\begin{itemize}
\item Les parenthèses autour de la condition ne sont pas nécessaires,
et sont en général omises dans la doc d'Apple.
\item Les accolades ne peuvent par contre pas être omises,
même s'il n'y a qu'une instruction à exécuter
\footnote{Apple à retenu la leçon, cherchez \og goto fail \fg{} dans Google.}.
\end{itemize}
\subsection{Et dans le cas contraire ?}
Évidement, maintenant que l'on a testé la valeur de $b$, il faut n'exécuter la division que si $b$ est non nul.
La solution que vous pourriez envisager est de faire ensuite
\mintinline{swift}{if b != 0},
mais il existe une solution plus claire : \mintinline{swift}{else}.
\begin{listing}[h]
\begin{minted}[linenos]{swift}
let b = 0
if b == 0 {
    // Code à exécuter si la conditions est vérifié
    // Éventuellement plusieurs instructions.
    println("b est nul ! Division impossible")
} else {
    println("a/b = \( a/b )")
    // Code exécuté dans le cas contraire.
}
// Code exécuté dans tout les cas.
\end{minted}
\caption{la syntaxe de \mintinline{swift}{if … else}}
\end{listing} %\)

Il est bien sûr possible d'imbriquer des \mintinline{swift}{if} à l'intérieur des accolades.
\subsection{Else if, exécuter des tests plus complexes}
Il est enfin possible de traiter des cas plus complexe avec \mintinline{swift}{else if}, << sinon si >>.

\begin{listing}[h]
\begin{minted}[linenos]{swift}
if condition1 {
    // Code à exécuter si 1 est vrai.
}
else if condition 2
{
    // Code à exécuter si 1 n'est pas vrai et que 2 est vrai.
} else if condition 3 {
    // Code à exécuter si 1 et 2 sont faux et 3 vrai .
}
/* … Avec autant de condition que nécessaires */
/* De façon optionnelle, un unique else, à exécuter si aucune des condition n'a été vérifiée. */
else {
    // Code à exécuter si aucune  condition n'est vérifiée.
}
// Code à exécuter dans tout les cas.
\end{minted}
\caption{\mintinline{swift}{if}, \mintinline{swift}{else if}, \mintinline{swift}{else}}
\end{listing}
\section{Quelques généralités sur les instructions composées}
Lorsque vous voyez apparaître de accolades, vous entrez dans un nouveau bloc de code, dans ce bloc, toutes les variables que vous déclarerez ne serons déclarées que dans ce bloc, et ne seront plus accessible une fois que vous en serez sorti. On parle de variable locales. Dans certain cas une variable locale est définie sur la ligne introduisant l'instruction composée.

On peut, par contre, accéder aux variables définies à l'extérieur de ce bloc, dite globales.

Il est possible, mais c'est déconseillé, de masquer une variable existant à l'extérieur du bloque en définissant une variable de même nom dans le bloc. On ne peut alors plus accéder qu'à la définition au sein du bloc.
\section{Application : une année est elle bissextile ?}
Après un début de chapitre quelque peu théorique, je vous propose un premier exercice un peu sérieux.
\subsection{Consignes}
Je vous demande d'écrire un code qui déclare une constante contenant une année
(que vous changerez pour tester le programme),
puis détermine si cette année est bissextile,
en affichant sur la sortie standard l'information.

Je rappelle à toute fin utiles qu'une année sur quatre est bissextile,
sauf pour les années qui sont un multiple de 100, qui ne sont pas bissextile,
sauf (encore une exception !) lorsque qu'elles sont multiples de 400.
\subsection{Indications}

Ceci est votre premier TP, ne vous inquiétez pas si ce n'est pas évident.

\begin{itemize}

\item N'oubliez pas l'assommant cours de méthodologie que je vous ai servi en introduction.

\item Pour savoir si un nombre est multiple d'un autre,
rien de mieux que le reste de la division euclidienne.

\item Un ou plusieurs \mintinline{swift}{if} (et éventuellement \mintinline{swift}{else}), comme vous voulez.
\end{itemize}
\pagebreak % Let's not cheat
\subsection{Corrections}
Je vous mets ici plusieurs corrections successives,
de la moins bonne à la meilleure, même si les 4 sont justes.

\begin{listing}[h!]
\begin{minted}[linenos]{swift}
let année = 2015
if année % 4 == 0 {
    if année % 100 == 0 {
        if année % 400 == 0 {
            println("Bissextile")
        } else {
            println("Pas bissextile")
        }
    } else {
        println("Bissextile")
    }
} else {
    println("Pas bissextile")
}
\end{minted}
\caption{Méthode bourrin.}
\end{listing}
Ce n'est pas la méthode la plus lisible,
même si c'est la plus évidente pour le débutant.

\begin{listing}[h!]
\begin{minted}[linenos]{swift}
let année = 2000 // Attention BUG ! ;)
if année % 400 == 0 {
    println("Bissextile")
} else if année % 100 == 0 {
    println("Pas bissextile")
} else if année % 4 == 0 {
    println("Bissextile")
} else {
    pintln("Pas bissextile")
}
\end{minted}
\caption{Un peu plus élégant.}
\end{listing}
C'est un peu moins brouillon, mais (comme précédemment), on duplique les lignes << bissextile >> et << pas bissextile >>, ce qui n'est pas génial. De plus, pour une année non multiple de 4 (a priori le plus fréquent), on effectue tous les tests alors que l'on pourrait conclure immédiatement.

\begin{listing}[h!]
\begin{minted}[linenos]{swift}
let année = 1995
if année % 4 != O {
    println("Pas bissextile")
} else if année % 100 != 0 {
    println("Bissextile")
} else if année % 400 != 0 {
    println("Pas bissextile")
} else {
    println("Bissextile")
}
\end{minted}
\caption{Méthode élégante optimisée.}
\end{listing}
Voilà qui est plus optimisé, mais il reste à résoudre la duplication de code :

\begin{listing}[h!]
\begin{minted}[linenos]{swift}
let année = 2042 // Attention, Geek !
if année % 4 != 0 || (année % 100 == 0 && année % 400 != 0) {
    println("Pas Bissextile")
} else {
    println("Bissextile")
}
\end{minted}
\caption{Méthode experte !}
\end{listing}

Je vous conseille de ne pas attaquer le chapitre suivant
si vous ne comprenez pas ces quatre corrections,
et sans avoir essayé les petits exercices faciles qui suivent.
\subsection{Variantes}
Essayez de refaire cela:
\begin{itemize}

\item en affichant l'année si ce n'est pas déjà fait ;
\item en stockant l'information dans un booléen ;
\item en n'obtenant cela sans \mintinline{swift}{if}, juste avec un grosse condition.

\end{itemize}
\section{Effectuer un disjonction de cas : switch … case}
Il existe une deuxième syntaxe permettant de créer des branchements :
\mintinline{swift}{switch … case}.
Il s'agit de choisir un bloc à exécuter selon la valeur d'une variable.
\paragraph{Attention :}Cette Syntaxe diffère de celle du C de façon importante.


En effet en Swift, il faut que tous les cas soient traités.
\begin{listing}[h]
\begin{minted}[linenos]{swift}
switch(<variable>){
case <valeur1> :
    // Code exécuté pour la valeur 1
    // IL ne peut pas y avoir de code vide.
case <valeur2>, <valeur3>:
    // Il est possible de lister plusieurs valeurs.
    // Code exécuté pour les valeurs 2 et 3
default :
    // Code exécuté pour tous les autres cas.
    // Il peut être omis s'il y a un nombre fini de cas.

}
\end{minted}
\caption{Syntaxe basique de \mintinline{swift}{switch}}
\end{listing}
\paragraph{Un cas vide}
Je détaillerai le rôle de \mintinline{swift}{break} par la suite, mais sachez que pour qu'un cas vide accepte de compiler, il suffit de mettre l'instruction \mintinline{swift}{break} là ou le code est attendu.


\section*{Conclusion}
\phantomsection
\addcontentsline{toc}{section}{Conclusion}
Dans ce chapitre vous avez appris à utiliser des conditions pour effectuer des branchements conditionnels, avec deux structures, \mintinline{swift}{if} … \mintinline{swift}{else if} … \mintinline{swift}{else}, et \mintinline{swift}{switch} … \mintinline{swift}{case}.

Vous avez surtout été confronté à votre première séance de travaux pratiques (TP).

Je vous conseille de vérifier que vous avez bien tout compris et que vous êtes capable de ré-écrire le programme du TP sans regarder la solution, avant de poursuivre votre lecture.

\section*{Exercices}
\phantomsection
\addcontentsline{toc}{section}{Exercices}
Je vous ai déjà laissé quelques exercices après le TP.
Si vous souhaitez pratiquer plus vous pouvez essayer de faire un des exercices qui suivent.
\begin{itemize}
\item Un programme qui dit si un nombre est pair ou impair,
\emph{(bonus, faire cela avec un nombre à virgule)}, positif ou négatif.
\item Un programme qui dit si un nombre est paire ou impaire en utilisant un \mintinline{swift}{switch}.
\item Un programme qui dit si un nombre à virgule est un entier.
\end{itemize}

Je n'ai pas beaucoup d'exercices pratiques à vous proposer car le TP reprend assez bien toutes les notions déjà exposées.


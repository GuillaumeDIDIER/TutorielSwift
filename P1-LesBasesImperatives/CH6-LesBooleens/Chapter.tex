\chapter{La logique booléenne et le type Bool}
Dans ce chapitre, nous allons voir comment vérifier des conditions dans notre programme.

\section{Le type Bool}
Je vous ai mentionné à l'instant les conditions ;
l'heure est donc venu de vous introduire le type
"qui va bien" pour stocker et manipuler les conditions.
Une condition peut être ou non vérifié, elle peut donc prendre deux valeurs : Vrai ou Faux.
En Swift, le type Bool correspond à cela. Un Bool, du nom du mathématicien anglais du XIX\textsuperscript{e} George Boole,
inventeur de la logique Booléenne, peut prendre les valeurs \mintinline{swift}{true} (Vrai en anglais) et \mintinline{swift}{false}.
\begin{listing}
\begin{minted}[linenos]{swift}
// Avec des constantes, rappel d'un chapitre précédent.
let les_oranges_sont_orange = true
let les_navets_sont_délicieux = false
\end{minted}
\caption{Deux booléens}
\end{listing}
\section{Un peu de logique}
Je ne vous ferez pas un cours complet de logique, mais je vais tout de même vous montrer les trois opérateurs de bases,
du point de vu théorique, avant de vous les montrer en Swift.
Pour toute la suite $A$ et $B$ désignent des booléens.
\subsection{La négation}
On appelle $non(A)$, le booléen qui est de la valeur opposée à $A$.
\begin{table}[h]
\centering
\begin{tabu} spread \linewidth {|l|r|}
\hline
$A$ & $non(A)$ \\ \hline
Vrai & Faux \\ \hline
Faux & Vrai \\ \hline
\end{tabu}
\caption{La négation}
\end{table}
\subsection{La conjonction}
La conjonction que l'on note $A$ $et$ $B$, n'est vraie que si à la fois $A$ et $B$ sont toute les deux vrai.
\begin{table}[h]
\centering
\begin{tabu} spread \linewidth {|l|l|r|}
\hline
$A$ & $B$ & $A$ $et$ $B$ \\ \hline
Vrai & Vrai & Vrai \\ \hline
Vrai & Faux & Faux \\ \hline
Faux & Vrai & Faux \\ \hline
Faux & Faux & Faux \\ \hline
\end{tabu}
\caption{La conjonction}
\end{table}
\subsection{La disjonction}
La disjonction que l'on note $A$ $ou$ $B$, est vraie si au moins un des deux booléens $A$ et $B$ est vrai.
\begin{table}[h]
\centering
\begin{tabu} spread \linewidth {|l|l|r|}
\hline
$A$ & $B$ & $A$ $ou$ $B$ \\ \hline
Vrai & Vrai & Vrai \\ \hline
Vrai & Faux & Vrai \\ \hline
Faux & Vrai & Vrai \\ \hline
Faux & Faux & Faux \\ \hline
\end{tabu}
\caption{La disjonction}
\end{table}
\subsection{Les lois de Morgan}
Dans cette section, nous allons aborder la négation respective d'une conjonction et d'une disjonction.

On a non$(A$ et $B) = $non$(A)$ ou non$(B)$,
et non$(A$ ou $B) = $non$(A)$ et non$(B)$.

\subsection{Distributivité}
Comme avec certaines opérations mathématiques, il est possible de distribuer et de factoriser des expressions logiques.

La conjonction est \emph{distributive} par rapport à la disjonction :

$A$ et $(B$ ou $C) = (A$ et $B)$ ou $(A$ et $C)$

Et la disjonction est aussi distributive par rapport à la conjonction :

$A$ ou $(B$ et $C) = (A$ ou $B)$ et $(A$ ou $C)$

\section{Les opérateurs booléens en pratique}
\subsection{La négation}
La négation est représentée en Swift par l'opérateur (unaire préfix) \verb"!", qui se place immédiatement avant le booléen dont on veut la négation.
\begin{listing}[h]
\begin{minted}[linenos]{swift}
let le_plat_du_jour_est_dégoutant = true
// ou false si votre cantine vous gate.
let je_vais_apprécier_mon_repas = !le_plat_du_jour_est_dégoutant
\end{minted}
\caption{Négation}
\end{listing}
\subsection{La conjonction}
En Swift la conjonction (ET), s'exprime avec l'opérateur (binaire) \verb"&&", qui se place entre les deux booléens dont on souhaite obtenir la conjonction.

\begin{listing}[h]
\begin{minted}[linenos]{swift}
// à l'entré d'un complexe militaire par exemple.
let peut_rentrer = connait_le_mot_de_passe && a_des_papiers_en_ordre
\end{minted}
\caption{Conjonction}
\end{listing}
\subsection{La disjonction}
En Swift la disjonction (ET), s'exprime avec l'opérateur (binaire) \verb"||" (ce symbole s'obtient en appuyant sur \emph{Alt + Shift + L}), qui se place entre les deux booléens dont on souhaite obtenir la disjonction.

\begin{listing}[h]
\begin{minted}[linenos]{swift}
// Spécial dédicace à un auteur du Site du Zér0 !
// Pour ouvrir un compte en banque il faut être assez âgé ou avoir beaucoup d'argent !
let peut_ouvrir_un_compte_chez_Picsou_Banque = 
est_assez_vieux || est_très_riche
\end{minted}
\caption{Disjonction}
\end{listing}

\subsection{Les priorités}
La négation est prioritaire par rapport aux opération mathématiques, prioritaires par rapport à la conjonction qui est elle même prioritaire sur la disjonction. Les parenthèses sont toujours utilisable pour contrôler la priorité.

\emph{Comme cette dernière priorité n'est pas évidente, je vous recommande de toujours mettre des parenthèses, comme si rien n'était spécifié à propos des priorité entre conjonction et disjonction. C'est aussi ce qu'\emph{Apple} recommande.}
\section{Les tests}
<< \emph{C'est bien jolie les booléens, mais comment on en fabrique à partir d'autre chose ?}>>
Il faut savoir que Certains types peuvent se convertir en booléens, mais pas tous, mais il est heuresement possible de tester une condition sur deux objets et obtenir un booléens.
\subsection{Égalité et inégalité}
Pour tester si deux objets sont égaux (ont la même valeur),
on utilise \verb"==". Attention à bien mettre \emph{deux} signes égal, pour distinguer cela de l'affectation d'une valeur à une variable. Pour obtenir en obtenir la négation, l'opérateur \verb"!=" existe aussi. On a bien \mintinline{swift}{(a != b) == !(a == b)}

On peut, entre autre, l'appliquer à des entiers, des nombres à virgule et des chaînes de caractères.
\subsection{Relation d'ordre}
Pour un certain nombre d'objet, il y a du sens à les ordonner.
Je vous passe la définition mathématique qui correspond, mais pour les types entiers et les nombres à virgule flottante.
Ils sont au nombre de 4 :
\begin{itemize}
\item \verb">", strictement supérieur à,
\item \verb"<", strictement inférieur à,
\item \verb">=", supérieur ou égal,
\item \verb"<=", inférieur ou égal,
\end{itemize}

\begin{listing}[h]
\begin{minted}[linenos]{swift}
println("2 > 3 ? \(2 > 3)")
println("2.5 > 2.4 ? \(2.5 > 2.4)")
let trois /*: Int */ = 3
println("2.3 < 3 ? \(2.3 < Double(trois))") // Conversion necessaire.
println(" (5 * 3) >= 15 \((5 * 3) >= 15)")
println(" (5 * 3) <= 15 \((5 * 3) <= 15)")
// Un nombre qui est à la fois plus petit et plus grand qu'un autre lui est égal !

\end{minted}
\caption{Quelques tests}
\end{listing}%\) Stupid syntactic coloration !
\subsection{Priorités}
Tous les tests ont la même priorité
(et s'exécutent de gauche à droite),
qui est supérieurs à la conjonction (et à la disjonction),
mais inférieur aux opérateurs mathématiques.



J'ai laissé quelques tests de coté, moins intéressants pour vous dans l'immédiat, ils seront traités plus tard.
\section{L'évaluation raccourcie}
Pour optimiser les programmes, lorsque l'ordinateur peut conclure en ayant évalué que le premier opérande, il n'évalue pas le deuxième. C'est le cas pour \mintinline{swift}{A && B} si \mintinline{swift}{A == false},
et pour \mintinline{swift}{A || B} si \mintinline{swift}{A == true}.
Ce comportement est garanti par le langage.

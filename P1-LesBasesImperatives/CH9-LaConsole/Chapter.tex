\chapter{Votre premier projet Xcode, la Console}

\section{Au commencement était la console...}
Vous êtes habitués à avoir des fenêtres sur votre écran d'ordinateur, mais il n'en a pas toujours été ainsi : les premiers ordinateurs possédant un écran affichaient simplement des lignes de texte comme celles ci (Figure ?)

Avant de pouvoir créer des programmes avec des fenêtres, il faut avoir acquis de solides bases en programmation, nous allons donc commencer par écrire des programmes en Console.
\section{Créons notre premier projet}
\section{Demandons du texte à l'utilisateur}
Contrairement à certains langages, Swift ne possède pas de fonction simple permettant de demander du texte à la console.
Il faut en effet passer par un code, certes plus sûr, mais plus technique et complexe.
Je vous ai donc concocté quelques fonctions
pour que vous n'ayez pas à vous occuper de ces détails,
on y reviendra plus tard. % Ref troisième/quatrième partie ?
Le code de ces fonctions est disponibe \TSwiftUrl{https://github.com/GuillaumeDIDIER/SwiftInput}{ici}{SwiftInput} ainsi qu'un projet XCode déjà configuré.
Mais je vais vous montrer comment les ajouter à votre projet existant.
\subsection{Ajoutons un second fichier à notre projet}
Vous pouvez aussi télécharger le fichier que je vous ai concocté en suivant ce  \TSwiftUrl{https://raw.githubusercontent.com/GuillaumeDIDIER/SwiftInput/master/SwiftInput.swift}{lien direct}{SwiftInput.swift}


Quelque soit votre façon d'obtenir le fichier SwiftInput.swift, enregistrez le à coté de votre source swift principle (main.swift). (S'il est affiché dans le navigateur téléchargez le, en choisissant source de la page, et en veillant à ne pas rajouter d'extension après .swift)
%TODO, captures d'écran.
\section{Pour les autres types}
\section{Application : Réclamer un nombre entier premier, positif}

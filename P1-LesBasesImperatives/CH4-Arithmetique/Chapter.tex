\chapter{Utilisation des variables}

Dans cette partie nous allons utiliser de façon un peu plus intéressante ces variables.

\section{Une vraie calculatrice !}
Je l'ai déjà dit, mais un ordinateur n'est qu'une grosse calculatrice, et pour vous prouver que
je ne vous mène pas en bateau, nous allons nous en servir :
\subsection{Additionnons deux nombres}
Pour additionner deux nombres, on utilise l'opérateur plus, \verb"+", comme au primaire.
\begin{listing}
\begin{minted}[linenos]{swift}
let nombre1 = 64
let nombre2 = 36
let résultat = nombre1 + nombre2
// On initialise la constante résultat avec le résultat de l'addition des deux nombres.
let flottant = 1.5
let entier = 5
let resultat2 = flottant + Double(entier) // Conversion !
let pi = 3 + 0.141592 
// Ceci ne marche que parce que les nombres littéraux n'ont pas de type.
\end{minted}
\caption{Des additions.}
\end{listing}

Attention, toutefois, au mélange des types. Pour additionner un entier et un nombre à virgule flottante, il faut convertir l'un des deux pour que les deux aient le même type. 
\subsection{Soustraction et multiplication}
De même, les bonnes vieilles soustractions et multiplications fonctionnent, avec les mêmes règles de priorité des opérations, en utilisant respectivement les opérateurs \verb"-" et \verb"*".
L'opérateur \verb"-" peut aussi être utilisé seul devant une variable numérique pour en obtenir l'opposé.
Exemple.
\begin{listing}
\begin{minted}[linenos]{swift}
let nombre1 = 10 * 10
let nombre2 = 6 * 6
let résultat = nombre1 - nombre2
// Pythagore a encore frappé. 
let moins_cent= -nombre1
\end{minted}
\caption{Multiplications et soustractions.}
\end{listing}

\subsection{Division et modulo}
La division, vous connaissez, mais vous allez sûrement me demander ce que j'entends par \emph{modulo}.

Il s'agit en fait du nom que les programmeurs donnent au \emph{reste} de la division euclidienne.
\subsubsection{La division}

Commençons par la division.

Il est possible de diviser deux entiers pour obtenir un entier, et deux nombres à virgule pour obtenir un nombre à virgule, attention, les trois du même type, à vous de faire des conversions si besoin est : \mintinline{swift}{LeBonType(<var_du_mauvais_type>) + var_du_bon_type}.
\begin{listing}[h]
\begin{minted}[linenos]{swift}
let entier = 49
let flottant = 5.0
let quotient_entier = entier / Int(flottant)
// Division d'entiers : 9
let quotient_flottant = Double(entier) / flottant
// Division de flottants : 9.8
\end{minted}
\caption{Les deux divisions.}
\end{listing}

\subsubsection{Le modulo}

%TODO : Rewrite entirely !

Le modulo \mintinline{swift}{(a % b)} est défini tel que \mintinline{swift}{a = b * un_entier + (a % b)} ou \verb"un_entier" est le plus grand entier en valeur absolue (en ommetant le signe) tel que \verb"a % b" soit du signe de \verb"a". \verb"a % b", est donc, en valeur absolue (sans le signe), strictement plus petit que \verb"b".
(en supposant \verb"b", positif, mais s'il est négatif on a le même réultat que s'il était positif)

En Swift, contrairement à la plupart des langage, le modulo n'est pas réservé à deux entiers, il est aussi possible de l'appliquer à deux nombres à virgule.

\begin{listing}[h]
\begin{minted}[linenos]{swift}
let entier = 49
let flottant = 5.0
let reste_entier = entier % Int(flottant)
// Division d'entiers : 4
let reste_flottant = Double(entier) % 2.5
// Division de flottants : 1.5
\end{minted}
\caption{Le modulo, avec des entiers ou des flottants.}
\end{listing}

\subsection{Assignations augmentées}

%TODO : Check name.

J'avais aussi omis de vous le préciser, mais la qualité première du programmeur est la \emph{fainéantise}. En fait je plaisante un peu, mais les programmeurs apprécient dès qu'ils ont moins à taper.
Par conséquent, au lieu d'écrire \mintinline{swift}{a = a + 1},
ils utilisent une notation raccourcie spéciale \mintinline{swift}{a += 1}.

On appelle cela des opérateurs \emph{augmentés}. Ils existent pour les cinq opérateurs que nous avons déjà vus :

\begin{longtabu} to \linewidth {|X[,l,m]|X[,l,m]|}
\hline Opérateur augmenté & Opérateur normal \\ \hline
\endhead
\mintinline{swift}{a += 1} & \mintinline{swift}{a = a + 1} \\ \hline
\mintinline{swift}{a -= 1} & \mintinline{swift}{a = a - 1} \\ \hline
\mintinline{swift}{a *= 2} & \mintinline{swift}{a = a * 2} \\ \hline
\mintinline{swift}{a /= 2} & \mintinline{swift}{a = a / 2} \\ \hline
\mintinline{swift}{a \%= 2} & \mintinline{swift}{a = a \% 2} \\ \hline
\caption{Les différents opérateurs augmentés.}
\end{longtabu}

\subsection{Incrémenter et décrémenter}
Comme les programmeurs sont vraiment très fainéants, il existe deux opérateurs pour ajouter et enlever 1 à une variable : \mintinline{swift}{++} et \mintinline{swift}{--}, formes raccourcies respectives de \mintinline{swift}{+=1} et \mintinline{swift}{-=1}. Ils peuvent être placés avant ou après une variable, avec une légère différence en terme de priorité, dont nous parlerons dans le chapitre dédié aux opérateurs avancés.
\subsection{Les priorités}
Si vous mélangez les différentes opérations, les opérateurs normaux seront exécutés en suivant les priorités mathématiques, c'est-à-dire les multiplications, divisions et (modulo) avant les additions et soustractions ; et les assignations (éventuellement augmentées) en dernier.
\emph{Pour modifier cette ordre on utilise des parenthèses, comme dans une expression mathématique.}
\section{Première approche des fonctions}
Vous vous souvenez de la ligne affichant du texte, au chapitre 2 ? Je vais ici vous l'expliquer un peu plus en détail.
\subsection{Une fonction}
Pour les matheux, \og Une fonction associe à tout élément d'un ensemble de départ au plus un élément d'un ensemble d'arrivé \fg{}.

Pour le programmeur, une fonction, c'est donc une sorte de \emph{boite noire}, à laquelle on donne à manger des paramètres, \emph{de types bien définis}, et qui renvoie une valeur (ou rien du tout).

Une fonction sert à réutiliser un bout de code sans avoir à le recopier à chaque fois, et permet de ne pas avoir à savoir comment cela fonctionne.
On reviendra sur l'utilité du découpage du code dans un chapitre ultérieur,
ainsi que sur un usage plus avancé des fonctions.

L'exemple que vous avez déjà vu est la fonction \mintinline{swift}{println} (on peut aussi utiliser \mintinline{swift}{print}, mais le retour à la ligne ne sera pas inclus de base), qui affiche une ligne de texte. Les conversions de type ressemblent aussi à des fonctions.

\begin{listing}[h]
\begin{minted}[linenos]{swift}
let entier = 42
let flottant = Double(entier)
println("Bonjour tout le monde")
\end{minted}
\caption{Deux appels de fonctions.}
\end{listing}

Un appel de fonction se présente sous la forme suivante.
\begin{listing}[h]
\begin{minted}[linenos]{swift}
nom_de_la_fonction(<paramètre>[, <paramètre2 …]) 
\end{minted}
\caption{Forme générale d'un appel de fonction.}
\end{listing}

Ici les noms correspondants sont \mintinline{swift}{Double} (le nom du type, en effet …) et \mintinline{swift}{println}. La deuxième ne renvoie rien, tandis que la première renvoie un \mintinline{swift}{Double} (comme c'est original …), que l'on récupère ici dans une variable, mais qui aurait pu être utilisé dans une autre expression, comme paramètre d'une autre fonction, ou dans une opération.

\subsection{Les Fonctions mathématiques}

Pour ceux qui connaissent un peu les maths, je vais ici vous donner la liste des fonctions mathématiques disponibles dans Swift. Attention, ce ne sont pas de vraies fonctions au sens mathématiques, donc pas de dérivation, intégration, limite et autres.
Je ne détail pas toutes les fonctions, qui sont pour la plupart directement issue de celle disponible en C/C++.
Si vous êtes intéressés par certaines, une simple recherche Google (ou autre) avec le nom de la fonction vous fournira les détails.
\begin{description}
\item[ceil, floor, round, trunc]
: Arrondis respectivement, au nombre entier supérieur, inférieur, le plus proche et vers 0. Prennent un flottant (Double ou Float), et renvoient un nombre du \emph{même type}.
\item[abs, fabs]
: cette bonne vieille fonction valeur absolue (Int ou Double ou Float), renvoie un nombre positif égal à plus ou moins le paramètre en entrée.
\item[sqrt, cbrt]
: racines carrée et cubique.
\item[pow]
: la fonction puissance, prend deux paramètre Double ou Float $x$ et $a$, et renvoie $x^{a}$. % Penser à faire un TP Opérateur **
\item[exp, exp2, expm1]
: exponentielles d'un Double ou d'un Float.
\item[log, log10, log1p, log2, logb]
: fonctions logarithmes, de Doubles et Floats.
\item[cos, cosh, sin, sinh, tan, tanh, acos, acosh, asin, asinh, atan, atanh, atan2]
: les fonctions trigonométriques. Elles sont nombreuses et je ne les détaillerai pas ici.
\item[erf, erfc, tgamma]
: fonctions mathématiques particulières.
\end{description}
Cette description sera peut-être encore étoffée.
\section*{Conclusion}
\phantomsection
\addcontentsline{toc}{section}{Conclusion}
Dans cette partie vous avez appris :
\begin{itemize}
\item à faire calculer votre ordinateur ;
\item ce qu'est le modulo pour un informaticien ;
\item ce que sont les priorités ;
\item ce qu'est (en gros) une fonction et comment l'utiliser.
\end{itemize}
\section*{Exercices}
\phantomsection
\addcontentsline{toc}{section}{Exercices}
\subsection*{Des heures, minutes, secondes vers les secondes}
Donnez-vous trois variables contenant des heures des minutes et des secondes
et faites calculer à votre ordinateur le nombre de secondes que cela fait.


\subsection*{Des secondes vers les heures, minutes, secondes}
Essayez d'écrire un code qui à partir d'une durée en seconde la transforme en (jours,) heures, minutes et secondes, soit l'inverse de l'exercice précédent.

%\section*{Correction}
%\phantomsection
%\addcontentsline{toc}{section}{Correction}

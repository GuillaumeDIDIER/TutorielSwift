\chapter{Les Boucles}
Dans ce chapitre, nous allons voir comment
répéter des instructions un nombre variable de fois.
\section{La boucle While}
\section{Énumérons des entiers, la boucle for in}
\section{Seconde syntaxe de la boucle for}
\section{Application : Copions des lignes}
\subsection{Consignes}
Votre but est d'écrire un programme qui
écrive un certain nombre de fois fixé dans une constante $n$ une ligne telle que << \emph{Je ne lancerai pas d'avions en papier.} >>.
\pagebreak % Let's not cheat.
\subsection{Indications}
Il vous faut une boucle, mais de quel type ?

Stocker la ligne et le nombre de fois dans deux constantes.

...\ est votre ami.
\pagebreak % Let's not cheat.
\subsection{Correction}
Posez vos stylos !
\begin{listing}[h]
\begin{minted}[linenos]{swift}
let n : UInt = 50 // Je ne suis pas complètement fou.
let punition : String = "Je dois chercher mes exercices !"
// Je ne vise personne ...
for i in 1...n { // n inclus
    println("\(i). \(punition)")
}
\end{minted}
\caption{Correction des lignes copiées.}
\end{listing}%\) Against stupid syntactique coloration.

\chapter{Les Conditions}
Dans ce chapitre, nous allons voir comment vérifier des conditions dans notre programme et
comment réagir différemment selon que cette condition est ou non vérifié.
\section{La logique booléenne et le type Bool}
Je vous ai mentionné à l'instant les conditions ; l'heure est donc venu de vous introduire le
type "qui va bien" pour stocker et manipuler les conditions.
\subsection{Le type Bool}
Une condition peut être ou non vérifié, elle peut donc prendre deux valeurs : Vrai ou Faux.
En Swift, le type Bool correspond à cela. Un Bool, du nom du mathématicien Boole,
inventeur de la logique Booléenne, peut prendre les valeurs true (Vrai en anglais) et false.
(Listing ?)
\subsection{Un peu de logique}
\subsection{Les opérateurs booléens en pratique}
\subsection{L'évaluation raccourcie}
\subsection{Les tests}
\section{Si ... Alors... Sinon ...}
\subsection{Exécuter des instructions uniquement lorsque une condition est vérifiée}

\subsection{Et dans le cas contraire ?}
\subsection{Else if, exécuter des tests plus complexes}
\section{Application : Une année est elle bissextile ?}
\subsection{Consignes}
\subsection{Correction}
\subsection{Variantes}

\chapter{Les Conditions}
Dans ce chapitre, nous allons voir comment vérifier des conditions dans notre programme et
comment réagir différemment selon que cette condition est ou non vérifié.
\section{La logique booléenne et le type Bool}
Je vous ai mentionné à l'instant les conditions ; l'heure est donc venu de vous introduire le
type "qui va bien" pour stocker et manipuler les conditions.
\subsection{Le type Bool}
Une condition peut être ou non vérifié, elle peut donc prendre deux valeurs : Vrai ou Faux.
En Swift, le type Bool correspond à cela. Un Bool, du nom du mathématicien anglais du XIX\textsuperscript{e} George Boole,
inventeur de la logique Booléenne, peut prendre les valeurs \mintinline{swift}{true} (Vrai en anglais) et \mintinline{swift}{false}.
\begin{listing}
\begin{minted}[linenos]{swift}
// Avec des constantes, rappel d'un chapitre précédent.
let les_oranges_sont_orange = true
let les_navets_sont_délicieux = false
\end{minted}
\caption{Deux booléens}
\end{listing}
\subsection{Un peu de logique}
Je ne vous ferez pas un cours complet de logique, mais je vais tout de même vous montrer les trois opérateurs de bases,
du point de vu théorique, avant de vous les montrer en Swift.
Pour toute la suite $A$ et $B$ désignent des booléens.
\subsubsection{La négation}
On appelle $non(A)$, le booléen qui est de la valeur opposée à $A$.
\begin{table}[h]
\centering
\begin{tabu} spread \linewidth {|l|r|}
\hline
$A$ & $non(A)$ \\ \hline
Vrai & Faux \\ \hline
Faux & Vrai \\ \hline
\end{tabu}
\caption{La négation}
\end{table}
\subsubsection{La conjonction}
La conjonction que l'on note $A$ $et$ $B$, n'est vraie que si à la fois $A$ et $B$ sont toute les deux vrai.
\begin{table}[h]
\centering
\begin{tabu} spread \linewidth {|l|l|r|}
\hline
$A$ & $B$ & $A$ $et$ $B$ \\ \hline
Vrai & Vrai & Vrai \\ \hline
Vrai & Faux & Faux \\ \hline
Faux & Vrai & Faux \\ \hline
Faux & Faux & Faux \\ \hline
\end{tabu}
\caption{La conjonction}
\end{table}
\subsubsection{La disjonction}
La disjonction que l'on note $A$ $ou$ $B$, est vraie si au moins un des deux booléens $A$ et $B$ est vrai.
\begin{table}[h]
\centering
\begin{tabu} spread \linewidth {|l|l|r|}
\hline
$A$ & $B$ & $A$ $ou$ $B$ \\ \hline
Vrai & Vrai & Vrai \\ \hline
Vrai & Faux & Vrai \\ \hline
Faux & Vrai & Vrai \\ \hline
Faux & Faux & Faux \\ \hline
\end{tabu}
\caption{La disjonction}
\end{table}
\subsubsection{Les lois de Morgan}
Dans cette section, nous allons aborder la négation respective d'une conjonction et d'une disjonction.

On a non$(A$ et $B) = $non$(A)$ ou non$(B)$,
et non$(A$ ou $B) = $non$(A)$ et non$(B)$.

\subsubsection{Distributivité}
Comme avec certaines opérations mathématiques, il est possible de distribuer et de factoriser des expressions logiques.

La conjonction est \emph{distributive} par rapport à la disjonction :

$A$ et $(B$ ou $C) = $(A$ et $B)$ ou $(A$ et $C)$

Et la disjonction est aussi distributive par rapport à la conjonction :

$A$ ou $(B$ et $C) = $(A$ ou $B)$ et $(A$ ou $C)$

\subsection{Les opérateurs booléens en pratique}
\subsubsection{La négation}
\subsubsection{La conjonction}
\subsubsection{La disjonction}
\subsubsection{Les priorités}
La négation est prioritaire par rapport aux opération mathématiques, prioritaires par rapport à la conjonction qui est elle même prioritaire sur la disjonction. Les parenthèses sont toujours utilisable pour contrôler la priorité.

\emph{Comme cette dernière priorité n'est pas évidente, je vous recommande de toujours mettre des parenthèses, comme si rien n'était spécifié à propos des priorité entre conjonction et disjonction. C'est aussi ce qu'\emph{Apple} recommande.}
\subsection{Les tests}
\subsubsection{Égalité et inégalité}

\subsubsection{Relation d'ordre}

J'ai laissé quelques tests de coté, ils seront traités plus tard.
\subsection{L'évaluation raccourcie}
Pour optimiser les programmes, lorsque l'ordinateur peut conclure en ayant évalué que le premier opérande, il n'évalue pas le deuxième. C'est le cas pour \mintinline{swift}{A && B} si \mintinline{swift}{A == false},
et pour \mintinline{swift}{A || B} si \mintinline{swift}{A == true}.
Ce comportement est garanti par le langage.
\section{Si ... Alors... Sinon ...}

\subsection{Exécuter des instructions uniquement lorsque une condition est vérifiée}

\subsection{Et dans le cas contraire ?}
\subsection{Else if, exécuter des tests plus complexes}
\section{Application : Une année est elle bissextile ?}
\subsection{Consignes}
\subsection{Correction}
\subsection{Variantes}

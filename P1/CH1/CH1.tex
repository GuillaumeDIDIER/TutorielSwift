\chapter{Avant de commencer}

\section{Comment lire ce cours}
\subsection{Pré requis}
Pour profiter pleinement de ce cours, il faut :
\begin{itemize}
\item Disposer d'un Mac (récent) tournant sous OS X Yosemite (10.10).
\item Être \emph{persévérant}, \emph{rigoureux} et \emph{logique}.
\item Ne pas faire une allergie à la langue de Shakespeare (On croirais entendre mourir un
auvergnat, je sais ;).
\item Un petit bagage mathématique, de préférence niveau troisième, mais un collégien ayant
acquis le primaire devrait s'en sortir en s’accrochant un peu.
\end{itemize}

\subsection{De le méthode}
Ce cours essaye d'être accessible au débutant, mais il est important d'acquérir de bonnes
bases pour pouvoir comprendre la suite. Certains chapitres peuvent être très denses,
donc n'hésitez pas à les relire plusieurs fois.
Pour bien acquérir un cours, il faut :
\begin{description}
\item[Le Comprendre] car on ne peut pas retenir et appliquer un cours sans le comprendre.
\item[Le Reprendre] plusieurs fois, car une bonne mémorisation ne peut se faire qu'à ce prix.
\item[Le Pratiquer], c'est à dire ne pas copier coller bêtement mon code sans le comprendre.
C'est en cherchant que l'on tire le meilleur profit d'un exercice : Tiens, j'ai essayé ça,
Pourquoi cela n'a il pas marché ? Essayez d'être curieux, de pousser mes exemples
dans leurs retranchements, de les modifier. Ne jetez pas l'éponge si vous n'avez pas
d'idée après 5 secondes ou si votre première idée ne marche pas. Enfin, essayez
d'exécuter pas à pas votre code dans votre tête pour vérifier qu'il fait bien ce que vous
aviez l'intention de faire.
\end{description}
Cette méthode de travail est la \emph{meilleure}, d'expérience d'élève de classe prépa (NB cette
méthode marche à tous les niveaux).
\subsection{Risque d’évolution}
Ce cours est susceptible d'évoluer rapidement du fait que Swift n'est pas encore stabilisé.

Le site https://github.com/ksm/SwiftInFlux/blob/master/README.md répertorie les
évolutions de Swift, y compris celles qui ne sont qu’envisagées.
\section{Quelques notions sur le fonctionnement d'un ordinateur}
\subsection{Qu'est-ce qu'un ordinateur ?}
Qu'est-ce qu'un ordinateur ? Avant tout une machine qui manipule des données (par
exemple la photo de votre chat, chien, petit-neveu ; le site Web que vous consultez, ou le
résultat d'une expérience du LHC), et cela selon un programme qui peut être changé,
c'est à dire que le programme est lui même une sorte de donnée.

Ainsi, les organes les plus importants d'un ordinateur sont sa mémoire et le processeur
qui exécute le programme, lui aussi stocké en mémoire.

Ensuite, un ordinateur possède aussi des entrées et des sorties, qui lui permettent de
communiquer avec le reste du monde (la carte graphique qui affiche ce texte à l'écran, la
carte réseau qui permet de charger des pages Web, la carte son, un clavier, une souris...).

Distinction supplémentaire : Un ordinateur possède 2 grands types de mémoire, une
mémoire qui ne s'efface pas si on coupe le courant, mais qui a généralement
l'inconvénient d'être lente, c'est par exemple le disque dur ou l'antique disquette ; et pour
pallier à cet inconvénient, un mémoire rapide, de moindre capacité et non persistante, la
mémoire vive ou RAM. Il y a aussi des mémoires dans lesquelles on ne peut que lire, qui
contiennent parfois les instructions nécessaire au démarrage de l'ordinateur.
\subsection{Qu'est-ce qu'un programme ?}
Un programme c'est la suite des instructions que doit exécuter le processeur
(e.g.\ additionner deux nombres, afficher bonjour à l'écran...).

Programmer, c'est créer un nouveau programme pour l'ordinateur.

Sauf qu'un ordinateur ne parle pas français, ni anglais, d'ailleurs ; Un ordinateur parle un
langage qui lui est propre, le \emph{binaire}, une suite de 0 et de 1, qui sont organisés en
instructions plutôt basiques, (par exemple écrit à tel emplacement dans la mémoire le
nombre 42, ou le résultat de l'addition de deux nombres en mémoire...). Ce n'est donc pas
raisonnable d'écrire un programme comme cela.
\section{Comment se faire comprendre d'un ordinateur ?}
\subsection{Un langage de programmation ?}
La première idée que l'on a eu c'est de mettre des mots un peu plus clair pour chaque
instruction, et d'utiliser un programme, l'\emph{assembleur}, pour effectuer la traduction. C'était
mieux, mais pour écrire du texte a l'écran et faire de jolis boutons, c'était encore trop
difficile.

C'est pour cela que l'on a inventé d'autres langages de programmation, de plus en plus
élaborés, qui sont ensuite traduits par un autre programme en code binaire.

L'un des langage ayant marqué l'informatique est le langage \emph{C}.

On classe donc les langages selon leur niveau d'abstraction par rapport au
fonctionnement de l'ordinateur. L'\emph{assembleur} est un langage de bas niveau, très (trop)
prohe du fonctionnement du processeur, tandis que le \emph{C} est de haut niveau (permettant
d'ordonner au processeur beaucoup de choses en une instruction), et le \emph{python},de très
haut niveau, pour ne citer que quelques exemples d'une très longue liste.%Liste Wiki

\emph{Swift}, le langage que nous allons étudier, est un langage de plus haut niveau
que le C.
\subsection{Swift}
Swift a été développé au sein d'Apple, sous l'égide de Chris Lattner à partir de 2010, et a
été présenté au monde en juillet 2014. Ce langage se veut << \emph{l'Objective-C sans le C} >>.

L'\emph{Objective-C} était le langage utilisé par Apple jusque là, mais qui était une extension du
C, ce qui impliquait que langage restait limité par la gestion de mémoire du C délicate et propre à causer de erreurs et failles de sécurités.

Le but était de créer un langage puissant, rapide, et sécurisé, mais avec une syntaxe
aussi agréable à utiliser qu’un langage de script.
\section{Les outils du programmeur}
Si vous avez bien lu la partie précédente, vous devriez déjà en avoir identifié un.

Il s'agit (comme vous l'avez tous deviné) d'un programme qui traduit notre code en
langage de haut niveau, appelé \emph{code source}, en code binaire. On l'appelle le
\emph{compilateur}.


Le deuxième outil est celui qui permet d'éditer le code (les plus malins m'auront vu venir).

Comme le code s'écrit dans des fichiers textes, un simple éditeur de texte, comme
TextEdit, pourrait suffire. Cependant les programmeurs utilisent en général des éditeurs
dédiés qui possèdent des fonctionnalités supplémentaires, comme la coloration du code.
% (cf Illustration en page suivante, il s’agit de l’éditeur TextWrangler)

Enfin un troisième outil est un debugger, qui permet de contrôler l'exécution pas à pas du
programme, pour comprendre pourquoi ça ne marche pas. (En théorie, si l'on ne faisait
pas d'erreur on pourrait s'en passer, mais << \emph{errare humanum est.} >> )

Mais pour simplifier les choses, il a été créé un type de programme particulier, les \emph{IDE,
Environnements de Développement Intégrés}, qui regroupent ces trois fonctionnalités. Sur
Mac OS X, cet EDI s'appelle \emph{Xcode}, et nous l'allons installer tout à l'heure !
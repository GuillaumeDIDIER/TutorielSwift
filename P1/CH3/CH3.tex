\chapter{Les variables, les Constantes, un peu d'arithmétique}
Comme je l'ai dit, un ordinateur manipule des données. Il faut donc pouvoir les stocker
quelque part.
\section{La mémoire d'un ordinateur, qu'est-ce qu'une \emph{variable} ?}
L'unité de base de la mémoire d'un ordinateur est \emph{le} bit (pas la !), qui possède deux états,
1 ou 0, à comparer aux 10 chiffres de notre système décimal.

Pour stocker des nombres
plus grand, il faut mettre plusieurs bits cote à cote, comme nous mettons plusieurs chiffres
pour écrire 1999, par exemple. Dans ce cas, le dernier chiffre correspond aux unités, l'avant dernier à
des paquets de 2, l'antépénultième 4, ... 8, 16, 32, 64... .

Les bits sont regroupés en mots
appelés Bytes en Anglais, d'une taille fixée par l'architecture du processeur. En Français
on traduit improprement ce terme par octet, qui désigne un mot de 8 bit qui peut prendre
\begin{math}2*2*2...*2\end{math} huit fois, noté \begin{math} 2^{8} \end{math}. Soit 256 valeurs (retenez ce nombre, on le retrouvera par la suite.)
Toute la mémoire est découpé en byte. On manipule forcément un ensemble de byte. (1,
2, 4, ...)
L'ordinateur peut aussi stocker des nombres binaires à virgule, comme nos nombres à virgule, avec les limitations de ces derniers. % Fixme Je ne détaille pas ici toute la représentation, si vous voulez des details je vous renvoie à cetv excellent tutoriel \url{}

\emph{Mais il n'y a que des nombres, comment on stocke du texte ?!}
Désolé, mais pour l'ordinateur, tout est un nombre, tout dépend de la façon dont on
l'interprète, on y viendra un peu plus tard.
\section{Demander poliment de la place pour stocker un nombre}

\section{Variables et Constantes}
On peut en fait imposer que la valeur d'une variable ne change plus après sa création :
c'est alors une constante. Elle se déclare comme une variable excepté que l'on utilise le
mot clé let, et que sa valeur doit être fixé à la déclaration.
Du fait de cette déclaration, le compilateur peut d'une part vous prémunir d'une erreur, et
d'autre part réaliser des optimisations supplémentaires. Apple vous recommande d'utiliser
des constantes en priorité, et de ne déclarer comme variable que ce qui doit l'être.
\section{Une vraie calculatrice !}
Je l'ai déjà dit, mais un ordinateur est une véritable calculatrice, et pour vous prouver que
je ne vous mène pas en bateau, nous allons nous en servir :
\section{Première approche des fonctions}